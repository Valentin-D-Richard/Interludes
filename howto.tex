\documentclass[11pt]{article}

% Gestion de l'encodage des accents et de la langue
%---------------------------
\usepackage[french]{babel}
\usepackage[utf8]{inputenc}
\usepackage[T1]{fontenc}
%---------------------------


% Structure de la page
%---------------------------
\usepackage[left=2cm, right=2cm, top=2cm, bottom=2cm]{geometry}
% \usepackage[babel=true,kerning=true]{microtype} % useless
%---------------------------


% Mise en page
%---------------------------
\pagestyle{plain}
\frenchbsetup % itemize avec puces
{
StandardItemLabels=true,
ItemLabels=\ding{43},
}
%---------------------------


% Utilisation des couleurs pour les titres
%---------------------------
\usepackage[dvipsnames]{xcolor}
\definecolor{gris125}{gray}{0.875}
\definecolor{gris25}{gray}{0.75}
\definecolor{gris5}{gray}{0.5}
\definecolor{blanc}{gray}{0}
%---------------------------


% Polices et caractères spéciaux
%---------------------------
\usepackage{lmodern}
\usepackage[sfdefault]{overlock}
\usepackage{calrsfs}

\def \epsilon{\varepsilon}
\def \phi{\varphi}
%---------------------------


% Écriture des mathématiques
%---------------------------
\usepackage{amsmath, amsfonts, amssymb, mathrsfs, esint}
\usepackage{amsthm}
\usepackage{wasysym}
\usepackage{marvosym} % \EUR
\usepackage{xspace}
\newcommand{\euro}{\EUR\xspace}
%---------------------------

% Affichage des tableaux
%---------------------------
\usepackage{booktabs}
\renewcommand{\arraystretch}{1.4}
%---------------------------


% Divers
%---------------------------
\usepackage{fancybox}
\usepackage{enumerate}
\usepackage{multicol}
\setlength\multicolsep{\parskip}
\usepackage{manfnt}
\usepackage{chemfig}
\usepackage{tabto}
\usepackage[inline]{asymptote}
\usepackage{wrapfig}
\usepackage{tikz}
\usetikzlibrary{mindmap,backgrounds}
%---------------------------


% Présentation des chapitres et sections
%---------------------------
\usepackage{hyperref}
\hypersetup{
colorlinks=true, %colorise les liens
breaklinks=true, %permet le retour à la ligne dans les liens trop longs
urlcolor= blue, %couleur des hyperliens
linkcolor= black,	%couleur des liens internes
citecolor=black	%couleur des références
}

% Sections
\usepackage[explicit, nobottomtitles]{titlesec}

\newlength\sectionboxwidth
\setlength\sectionboxwidth{\textwidth}
\addtolength\sectionboxwidth{-2\fboxsep}

\DeclareFixedFont{\sectionfont}{T1}{phv}{bx}{n}{11mm}
\titlespacing*{\section}{0pt}{-1\baselineskip minus \parskip}{2\baselineskip}
\renewcommand\thesection{\Roman{section}}

\titleformat{\section}[display]{\LARGE\normalfont\bfseries}{}{0pt}{\hfill\thesection.~#1\hfill}

% Sous-sections
\renewcommand\thesubsection{\thesection.\arabic{subsection}}

\titleformat{\subsection}[hang]{}{}{0pt}{\colorbox{Tan!15!white}{\parbox[c][][s]{\sectionboxwidth}{\Large \bfseries \rule[-0.2cm]{0mm}{0.75cm} \thesubsection{} - #1}}}
\titlespacing*{\subsection}{0pt}{1.2\bigskipamount}{.8\bigskipamount}

% Sous-sous-sections
\titleformat{\subsubsection}{\large\normalfont\bfseries}{\\\qquad \thesubsubsection.}{0.75em}{\qquad \underline{\smash{#1}}}


% Limite de la table des matières
%---------------------------
\setcounter{secnumdepth}{2}
%---------------------------


% Création des encadrés personnalisés et des macros
%---------------------------
\usepackage{tcolorbox}
\tcbuselibrary{theorems}
\usepackage{varwidth}
\tcbuselibrary{skins}
\renewcommand{\emph}[1]{\textbf{\textcolor{RedViolet}{#1}}}
% création du type Attention

% création macros attention
\def \danger{\textdbend}
\newlength{\capletterheight}
\settoheight{\capletterheight}{A} % height of the capital A
\newcommand{\warning}{\protect\includegraphics[height = \capletterheight]{attention.png}~}

\newtcbtheorem[no counter]{Attention}{\warning Attention !}
%
{enhanced,frame empty,interior empty,colframe=RedViolet,
	coltitle=black,fonttitle=\bfseries,colbacktitle=RedViolet!50!white,
	borderline={0.5mm}{0mm}{RedViolet},
	attach boxed title to top left={yshift=-2mm, xshift=10mm},
	boxed title style={boxrule=0.4pt},varwidth boxed title}{theo}

% création du type Bon à savoir / faire

\newenvironment{Asavoir}[1]{
\tcbox[colback=RoyalBlue!15!white,colframe=RoyalBlue,nobeforeafter,boxsep=0pt,on line,boxrule=0.4pt]{
\textbf{\!\!Bon à savoir\!\!}}~#1}{}
	
\newtcbtheorem[no counter]{Afaire}{Bon à faire}
%
{enhanced,frame empty,interior empty,colframe=RoyalBlue,
	coltitle=black,fonttitle=\bfseries,colbacktitle=RoyalBlue!15!white,
	borderline={0.5mm}{0mm}{RoyalBlue},
	attach boxed title to top left={yshift=-2mm, xshift=10mm},
	boxed title style={boxrule=0.4pt},varwidth boxed title}{theo}

% création du type À anticiper

\newtcbtheorem[no counter]{Aanticiper}{À anticiper}
%
{enhanced,frame empty,interior empty,colframe=Brown,
	coltitle=black,fonttitle=\bfseries,colbacktitle=Brown!15!white,
	borderline={0.5mm}{0mm}{Brown},
	attach boxed title to top left={yshift=-2mm, xshift=10mm},
	boxed title style={boxrule=0.4pt},varwidth boxed title}{theo}

%---------------------------
