 \documentclass[11pt]{article}

% Gestion de l'encodage des accents et de la langue
%---------------------------
\usepackage[french]{babel}
\usepackage[utf8]{inputenc}
\usepackage[T1]{fontenc}
%---------------------------


% Structure de la page
%---------------------------
\usepackage[left=2cm, right=2cm, top=2cm, bottom=2cm]{geometry}
% \usepackage[babel=true,kerning=true]{microtype} % useless
%---------------------------


% Mise en page
%---------------------------
\pagestyle{plain}
\frenchbsetup % itemize avec puces
{
StandardItemLabels=true,
ItemLabels=\ding{43},
}
%---------------------------


% Utilisation des couleurs pour les titres
%---------------------------
\usepackage[dvipsnames]{xcolor}
\definecolor{gris125}{gray}{0.875}
\definecolor{gris25}{gray}{0.75}
\definecolor{gris5}{gray}{0.5}
\definecolor{blanc}{gray}{0}
%---------------------------


% Polices et caractères spéciaux
%---------------------------
\usepackage{lmodern}
\usepackage[sfdefault]{overlock}
\usepackage{calrsfs}

\def \epsilon{\varepsilon}
\def \phi{\varphi}
%---------------------------


% Écriture des mathématiques
%---------------------------
\usepackage{amsmath, amsfonts, amssymb, mathrsfs, esint}
\usepackage{amsthm}
\usepackage{wasysym}
\usepackage{marvosym} % \EUR
\usepackage{xspace}
\newcommand{\euro}{\EUR\xspace}
%---------------------------

% Affichage des tableaux
%---------------------------
\usepackage{booktabs}
\renewcommand{\arraystretch}{1.4}
%---------------------------


% Divers
%---------------------------
\usepackage{fancybox}
\usepackage{enumerate}
\usepackage{multicol}
\setlength\multicolsep{\parskip}
\usepackage{manfnt}
\usepackage{chemfig}
\usepackage{tabto}
\usepackage[inline]{asymptote}
\usepackage{wrapfig}
\usepackage{tikz}
\usetikzlibrary{mindmap,backgrounds}
%---------------------------


% Présentation des chapitres et sections
%---------------------------
\usepackage{hyperref}
\hypersetup{
colorlinks=true, %colorise les liens
breaklinks=true, %permet le retour à la ligne dans les liens trop longs
urlcolor= blue, %couleur des hyperliens
linkcolor= black,	%couleur des liens internes
citecolor=black	%couleur des références
}

% Sections
\usepackage[explicit, nobottomtitles]{titlesec}

\newlength\sectionboxwidth
\setlength\sectionboxwidth{\textwidth}
\addtolength\sectionboxwidth{-2\fboxsep}

\DeclareFixedFont{\sectionfont}{T1}{phv}{bx}{n}{11mm}
\titlespacing*{\section}{0pt}{-1\baselineskip minus \parskip}{2\baselineskip}
\renewcommand\thesection{\Roman{section}}

\titleformat{\section}[display]{\LARGE\normalfont\bfseries}{}{0pt}{\hfill\thesection.~#1\hfill}

% Sous-sections
\renewcommand\thesubsection{\thesection.\arabic{subsection}}

\titleformat{\subsection}[hang]{}{}{0pt}{\colorbox{Tan!15!white}{\parbox[c][][s]{\sectionboxwidth}{\Large \bfseries \rule[-0.2cm]{0mm}{0.75cm} \thesubsection{} - #1}}}
\titlespacing*{\subsection}{0pt}{1.2\bigskipamount}{.8\bigskipamount}

% Sous-sous-sections
\titleformat{\subsubsection}{\large\normalfont\bfseries}{\\\qquad \thesubsubsection.}{0.75em}{\qquad \underline{\smash{#1}}}


% Limite de la table des matières
%---------------------------
\setcounter{secnumdepth}{2}
%---------------------------


% Création des encadrés personnalisés et des macros
%---------------------------
\usepackage{tcolorbox}
\tcbuselibrary{theorems}
\usepackage{varwidth}
\tcbuselibrary{skins}
\renewcommand{\emph}[1]{\textbf{\textcolor{RedViolet}{#1}}}
% création du type Attention

% création macros attention
\def \danger{\textdbend}
\newlength{\capletterheight}
\settoheight{\capletterheight}{A} % height of the capital A
\newcommand{\warning}{\protect\includegraphics[height = \capletterheight]{attention.png}~}

\newtcbtheorem[no counter]{Attention}{\warning Attention !}
%
{enhanced,frame empty,interior empty,colframe=RedViolet,
	coltitle=black,fonttitle=\bfseries,colbacktitle=RedViolet!50!white,
	borderline={0.5mm}{0mm}{RedViolet},
	attach boxed title to top left={yshift=-2mm, xshift=10mm},
	boxed title style={boxrule=0.4pt},varwidth boxed title}{theo}

% création du type Bon à savoir / faire

\newenvironment{Asavoir}[1]{
\tcbox[colback=RoyalBlue!15!white,colframe=RoyalBlue,nobeforeafter,boxsep=0pt,on line,boxrule=0.4pt]{
\textbf{\!\!Bon à savoir\!\!}}~#1}{}
	
\newtcbtheorem[no counter]{Afaire}{Bon à faire}
%
{enhanced,frame empty,interior empty,colframe=RoyalBlue,
	coltitle=black,fonttitle=\bfseries,colbacktitle=RoyalBlue!15!white,
	borderline={0.5mm}{0mm}{RoyalBlue},
	attach boxed title to top left={yshift=-2mm, xshift=10mm},
	boxed title style={boxrule=0.4pt},varwidth boxed title}{theo}

% création du type À anticiper

\newtcbtheorem[no counter]{Aanticiper}{À anticiper}
%
{enhanced,frame empty,interior empty,colframe=Brown,
	coltitle=black,fonttitle=\bfseries,colbacktitle=Brown!15!white,
	borderline={0.5mm}{0mm}{Brown},
	attach boxed title to top left={yshift=-2mm, xshift=10mm},
	boxed title style={boxrule=0.4pt},varwidth boxed title}{theo}

%---------------------------

\usepackage{multirow}
\usepackage{stmaryrd}

\usepackage{manfnt}
\usepackage{accents}

\usepackage[normalem]{ulem} % strikethrough text

\usepackage{color}
\newcommand{\Tino}[1]{\textit{\textcolor{teal}{[Tino]: #1}}}

\title{How-to :\texorpdfstring{\\}{, } Interludes}
\date{version mai 2020}
\author{Bureaux BdL des 4 ENS 2017-2020\\
Rédigé par Tino}

\begin{document}
\maketitle

\tableofcontents\bigskip
% Pour les personnes qui s'étonneraient de l'orthographe de certains mots, je suis la réforme de 1990, notamment : ile, chaine, paraitre ou encore aout sans accent circonflexe.

\textit{Quelques points de cette fiche sont tirés directement du bilan des Interludes 2019 et 2020.}

%%%%%
\subsection*{Description de l'évènement}

Les Interludes sont un évènement ludique organisé annuellement alternativement par une des quatre ENS de France pour
l’ensemble des élèves de ces écoles (environ 300 participant·e·s). Il se déroule sur trois jours, du vendredi soir au dimanche après-midi en continu, vers mi-février généralement. Cet événement a été mis en place pour la première fois à Cachan en 2013. La préparation d'un tel rassemblement demande une grande organisation.

Cette fiche how-to a pour vocation d'aider au mieux à s'y retrouver et penser à tout.

%%%%%%%%%%
\section{Préparation des Interludes}

%%%%%
\subsection{Équipe organisatrice}\label{sec:equipe}

Le premier ingrédient des Interludes est \emph{l'équipe} qui se chargera de mener à bien cette épreuve. Idéalement, des personnes peuvent se porter volontaires dès le début de l'année précédente. Il peut s'agir d'un club dont c'est la mission spéciale ou de volontaires 1A ou expérimentés. Le plus important est que ces personnes aient une bonne visions de ce à quoi ressemble en pratique cet événement, par exemple en allant aux Interludes précédentes. Voici quelques critères pour former une bonne équipe :
\begin{itemize}
    \item[++] Être suffisamment : \emph{au moins 6 personnes très actives}, et quelques autre prêtes à aider plus ponctuellement
    \item[+] \textbf{Être soudé·e·s} : dans les moments difficiles, et pour une bonne communication, mieux vaut être prêt·e·s à bien s'entraider
    \item Être complémentaires : pour pouvoir répartir les tâches optimalement
    \item Avoir un peu d'expérience d'organisation (au moins pour certain·e·s)
\end{itemize}

À la première réunion, priorité est donnée à la \emph{répartition des rôles}. Il est primordial de diviser les tâches. D'une part, cela évite certain·e·s d'être submergés entre ce volontariat, les cours et la vie personnelle. D'autre part, cela force les informations à circuler. Plus il y a de gens au courant des points clés, plus ceux-ci ont de chance d'être résolus et bien traités. Voici une liste de postes essentiels, certains peuvent être pris conjointement par une même personne :
\begin{itemize}
    \item \textbf{Président·e} : il ou elle  gère l'organisation globale, la coordination entre les tâches et les prises de décisions majeures. Cette personne est aussi chargée de représenter l'événement légalement et devant l'administration.
    \item \textbf{Trésorier·e} : il ou elle gère le budget total, la répartition des dépenses et des recettes
    \item \textbf{Responsable inscriptions} : il ou elle doit créer le formulaire d'inscription aux Interludes, gérer les demandes particulières, les modalités d'inscriptions et de paiement des autres ENS (avec trésorier·e et respo communication) et les arrivées
    \item \textbf{Responsable repas} : il ou elle prévoit les repas et s'assure de leur préparation
    \item \textbf{Responsable activités} : il ou elle se charge de créer le formulaire d'appel à projet, créer l'emploi du temps, inviter à l'inscription aux activités, répartir les demandes et prévenir les sélectionné·e·s et proposant·e·s.
    \item \textbf{Responsable communication} : il ou elle a pour mission de faire connaitre l'événement au sein de l'école d'accueil et ailleurs, via des affiches et les réseaux étudiants ou sociaux, et de s'occuper du site internet
    \item \textbf{Coordinateur·rice} : il ou elle a le rôle important d'être le relais entre les autres écoles et l'équipe organisatrice, par l'intermédiaire de groupes ou personnes responsables de transmettre les informations dans chaque autre ENS
    \item \textbf{Responsable installation et rangement} : il ou elle prévoit toutes les affaires à mettre en place et veille à ce que tout soit rangé et propre pour rendre à temps les locaux
    \item \textbf{Responsable sécurité} : il ou elle s'assure que tout est mis en place en amont et pendant pour éviter tout danger
    \item \textbf{Secrétaire} : il ou elle s'occupe de rendre compte des réunions
\end{itemize}

Il va sans dire que la communication au sein de l'équipe doit pouvoir être rapide et efficace. S'échanger les numéros de téléphone peut s'avérer primordial.

%%%%%
\subsection{Besoins mobiliers et immobiliers}

Les Interludes requièrent quelques salles proches et des équipement pour diverses fonctions :

$\bullet$ \emph{Une cuisine} pour préparer à manger, avec
\begin{itemize}
    \item un \textbf{lavabo}, voire deux pour faire la vaisselle
    \item un \textbf{grand espace de travail} pour éplucher et découper les aliments
    \item \textbf{de la place} pour ranger tous les ingrédients et préparations pour un weekend entier
    \item un \textbf{grand frigo}
    \item beaucoup de \textbf{grands plats} pour mélanger et cuire les plats
    \item un grand nombre d'\textbf{ustensiles} (très grandes cuillères, louches, épluche-légumes, grands couteaux,...)
    \item 2 à 3 \textbf{bouilloires}
    \item impérativement, des \textbf{plaques de cuisson}
    \item idéalement, un voire deux fours $\to$ Repas \S\ref{sec:repas}
\end{itemize}

$\bullet$ \emph{Une salle à manger} où accueillir environ au moins 100 personnes à la fois. Le temps de servir le repas, certain·e·s pourront avoir fini de manger et laisser leur place à d'autres. Il faut donc :
\begin{itemize}
    \item des tables
    \item des bancs ou des chaises pour environ 100 personnes
\end{itemize}

$\bullet$ \emph{Des salles pour les activités}. En comptant environ 30 activités prévues sur tout le weekend, il peut y en avoir jusqu'à une dizaine en simultané à un moment donné, typiquement le samedi après-midi et soir $\to$ Activités \S\ref{sec:activités}. En fonction des propositions, on peut imaginer en moyenne un besoin de :
\begin{itemize}
    \item \textbf{2 à 4 petites salles} : jdr plateau
    \item \textbf{3 à 6 salles moyennes} : huis-clos, jeux d'ambiance, jeux vidéos
    \item \textbf{1 à 2 grandes salles} (avec si possibles de petits espaces annexes, souvent demandés) : grands huis-clos, grandeurs nature
    \item 1 grande salle ou plusieurs salles moyennes en plus : \textbf{jeux de société}
    \item \textbf{1 amphithéâtre} avec projection possible : activités diverses, impro, chorale, amphi de fin
\end{itemize}
Le besoin mobilier est généralement faible, juste assez de chaises pour pouvoir faire asseoir tout le monde, ex. \textbf{une vingtaine de (doubles) tables à 4 à 6 places} pour l'espace jeux de sociétés.

$\bullet$ \emph{Un moyen d'hébergement}, par exemple un gymnase ou des salles vides $\to$ Logement \S\ref{sec:logement}.

Et idéalement :

$\bullet$ \textbf{Une salle fermable à clé} pour ranger les affaire personnelles $\to$ Sécurité \S\ref{sec:sécurité}.

$\bullet$ \textbf{Une salle de repos en journée}, pour celles et ceux qui n'ont pas beaucoup dormi de la nuit.

Il est bien sûr plus agréable si ces salles sont proches les unes des autres, où au moins dans un même complexe. Cela réduit notamment le plan à insérer sur le dépliant, mais aussi le fléchage vers les différents lieux. Malheureusement, la plus grand contrainte, ici, est souvent l'accord de l'administration $\to$ accord de l'administration \S\ref{sec:administration}.

%%%%%
\subsection{Sécurité}\label{sec:sécurité}

\emph{La sécurité} des biens et personnes est un élément à ne pas oublier, dont un plan sera sûrement attendu par l'administration. Les points à certifier sont les suivants :
\begin{itemize}
    \item Assurer la sûreté des personnes contre les potentiels heurts (violences physiques et morales) et leur intégrité (blessures, épuisement, \sout{chagrin d'amour})
    \item Prévenir les dégâts matériels et le vol
\end{itemize}

Les mesures traditionnellement prises sont les suivantes :
\begin{itemize}
    \item \textbf{Alcool interdit}
    \item \textbf{Système de bracelet} pour identifier les participant·e·s : \href{https://www.bracelets-evenementiels.com}{bracelets-evenementiels.com}
    \item \textbf{Numéro d'urgence} en cas d'incident
    \item Permanence d'un \textbf{poste de secours} (trousse premiers soins)
\end{itemize}

Et éventuellement :
\begin{itemize}
    \item Salle fermable à clé ou surveillée où ranger les affaires personnelles gérée par un·e permanencier·e
    \item Contrôle des salles après une activité
\end{itemize}

C'est au choix de l'équipe d'adapter ces mesures. Mais par exemple, pour l'interdiction de l'alcool, les gens ne s'attendent de toute façon pas à en trouver.

Au vu du grand nombre de participant·e·s, et surtout car ce sont des personnes externes, l'administration peut requérir le service de \textbf{vigiles}. Cela dépend du bâtiment en question, notamment s'il y a au non déjà des personnes en permanence les weekends. Pour 2 vigiles de 8h30 à 20h30 le samedi, compter \textbf{environ 600\euro}. Pour des horaires réguliers (9h - 19h), compter un peu moins. Avec 300 personnes prévues, cela revient à seulement 2\euro en plus par personne.

%%%%%
\subsection{Logement}\label{sec:logement}

\emph{Le logement} est une partie souvent délicate, où on ne peux que rarement assurer un grand confort. Il est coutume de demander aux participant·e d'\textbf{apporter un sac de couchage} voire un matelas. Ainsi, le plus ``simple'' est de réserver un espace en intérieur, loin des jeux, où les gens peuvent dormir la nuit. Il peut s'agir de salles de classes, où d'un gymnase, souvent plus froid mais équipé de douches. La possibilité de cette solution réside avant tout sur le fait que les personnes seront sobres et calmes. Il est impératif de charger quelqu'un de \textbf{surveiller ces salles}. C'est un requis de sécurité. L'astuce est de placer deux personnes à jouer à un jeu calme.

Si la location de salles est impossible, il faut demander l'aide des participant·e·s de l'école accueillante. Ces personnes doivent s'engager à héberger chez elles un·e ou plusieurs autres pour une ou deux nuits. Certain·e·s ont déjà des connaissances, et s'arrangent ensemble. Pour le reste, il est conseillé de mettre en place un système d'offre et de demande, éventuellement en prenant en compte les horaires prévues de coucher et de lever. Ces dernières dépendant notamment de l'emploi du temps. Il faut donc s'attendre à beaucoup de demandes de dernière minute. Ne pas hésiter à rabâcher les formulaires d'offre et de demande d'hébergement. Attention, l'attribution manuelle prend donc du temps sur les derniers jours. Mieux vaut s'assurer que les numéros de téléphone sont bien échangés entre hôtes et hébergés, pour éviter les mauvaises surprises (changement de plan, logé·e n'arrivant jamais,...). Malgré tout, on conseille de prévoir une salle de repos ``au cas où'' pour les personnes prises au dépourvu. Un nombre certain de personnes ne prévoient juste pas ce moment et demandent à la dernière minute ou trouvent une chaise à l'écart.

Si le logement est loin du point central, il est recommandé d'organiser un \textbf{pédibus à heures régulières} le soir, pour que personne ne s'égare ou ne tombe dans une mauvaise situation, surtout s'il s'agit de quitter l'établissement.

% Bastien : parler des problèmes de logements dus à un manque d'organisation des participants ?

%%%%%
\subsection{Accord de l'administration}\label{sec:administration}

Il est parfois extrêmement compliqué de faire comprendre que, contrairement à la plupart des grands évènements intra- ou inter-ENS, les Interludes ``ne coutent pas grand chose''. Ce que \emph{l'administration} a besoin de savoir avant tout est :
\begin{itemize}
    \item \textbf{Quand} a lieu l'événement ?
    \item \textbf{Qui} est concerné, combien de personnes prévues ?
    \item \textbf{Quels sont les demandes} : quelles salles, quelles heures d'utilisation précisément, quels équipements ?
\end{itemize}

Il faut que ce soit d'une limpidité absolue et le plus précis possible, comme s'ils n'avaient plus qu'à cocher.

Le principal frein est la responsabilité que prend l'école à propos de l'ensemble des personnes présentes sur son espace. S'il arrive le moindre bémol, elle est en charge. Il est donc de son ressort de s'assurer qu'elle est en mesure d'assurer la sécurité de l'événement, même si vous avez prévu votre propre plan.

\begin{Afaire}{}{}
Il est important de rappeler, même si c'est évident et chiant, qu'aux Interludes :
\begin{itemize}
    \item \textbf{l'alcool est interdit}
    \item \textbf{les salles ne sont pas décorées ni vidées de leur mobilier}
    \item \textbf{les prises électriques ne sont pas utilisées} (sauf salle jeux vidéos)
    \item \textbf{l'accès est contrôlé}  (grâce aux bracelets)
\end{itemize}
Ces petits point leur permettent d'évaluer si un dossier (aux pompiers, à la préfecture) est nécessaire.
\end{Afaire}

La demande est à envoyer le plus tôt possible, par exemple en avril ou mai de l'année précédente. Cependant, l'école n'a peut-être pas encore fixé son emploi du temps si longtemps à l'avance, donc peut vous faire patienter. Dans ce cas, vous pouvez essayer de demander plutôt à quel moment l'administration sera plus en mesure de savoir si l'événement est organisable.

\begin{Attention}{}{}
Comme disait Platon, \emph{un entretien vaut mieux que mille courriels}. Relancer est souvent vain. Allez directement voir la personne responsable de la vie étudiante (ou équivalent) pour lui demander de faire passer l'information. Assurez-vous que l'information soit bien passée et arrivée aux oreilles des responsables de la sécurité, puis allez directement leur parler.
\end{Attention}

\begin{Attention}{}{}
Ce genre de processus est très facilement irritant. Il vaut mieux garder un bon contact avec l'administration. Le petit conseil de Tino : venez par deux si possible et faites relire vos mails avant de les envoyer, au cas où ils seraient non volontairement d'un mauvais ton.
% Tino : Bon, c'est peut-être un peu trop expérience personnelle pour y figurer.
% Corto : C'est pertinent au moins pour les Interludes 2023
\end{Attention}

%%%%%
\subsection{Communication}

Si votre événement vient d'être accepté par l'administration, félicitations ! Maintenant il est temps de \emph{le faire savoir}. Très en amont, par exemple \textbf{dès septembre}, la date peut être communiquée aux autres ENS. Il peut y avoir des conflits avec d'autres événements selon les écoles, donc prévenir tôt peut amener envisager un changement de date. Plus la date est assurée par l'administration et communiquée tôt, plus les autres ENS auront de temps pour s'organiser (réservation du bus, formulaire d'inscription au bus,...).

\begin{Asavoir}
Il est apprécié de faire attention à ne pas placer les Interludes le weekend entre les deux semaines des \textbf{vacances de février}. Mais attention, d'autres évènements (gala,...) envient aussi ces dates après les examens de janvier et avant les départs en stage long.
\end{Asavoir}

%%%
\subsubsection*{En interne}

Au sein de l'école concernée, il faut de plus inviter les intéressé·e·s à réserver cette date pour avoir de l'aide. Notamment, l'installation, les diverses permanences et le rangement nécessite beaucoup de \textbf{petits bras}.

Mais c'est aussi l'occasion de recruter le plus de personnes non habituées à ce genre de fêtes. Cela passe par une annonce dans les médias de l'école (journal étudiant, mailing-list événement, événement Facebook\footnote{Le lien peut avoir changé.} \href{https://www.facebook.com/Interludes2020/}{facebook.com/Interludes20XX},  affiche,...).

%%%
\subsubsection*{En externe}

La \emph{communication avec les autres écoles} se fait via l'intermédiaire de responsables chargé·e·s de communiquer à leur école respective les informations transmises par l'équipe organisatrice. Il s'agit souvent des bureaux des associations de jeux. Le coordinateur ou la coordinatrice doit tenir cette relation et transmettre les questions en retour. Il faut s'assurer des points suivants :
\begin{itemize}
    \item \textbf{Dates} et conditions de l'événement transmises
    \item \textbf{Formulaires} d'inscription et d'appel à projet transmis
    \item Bus / trains réservés, \textbf{heures d'arrivée et de départ}
    \item \textbf{Besoin en sandwichs} ou non pour le retour
    \item Courriels d'annonce et formulaire d'inscription aux activités transmis
\end{itemize}

%%%
\subsubsection*{Informations}

Pour attirer les gens, \emph{une affiche} est réalisée. Elle contient au minimum ``Interludes'' avec la date, le nom de l'école et le prix d'entrée, éventuellement avec le prix des repas. Il est de tradition d'y faire paraitre un diablotin, comme emblème des Interludes. Vous pouvez trouver l'historique des affiches ainsi que les dates des précédentes versions ici : \href{https://wiki.crans.org/VieBdl/InterLudes}{wiki.crans.org/VieBdl/InterLudes}.

Il est aussi dans la coutume d'\emph{offrir une tasse} comme goodies: prévoir entre 250 et 300. Le logo principal vectorisé (par exemple avec \href{https://inkscape.org/fr/}{Inkscape}) ainsi que la date ou l'année sert à la personnaliser, pour en garder un long souvenir.

\begin{Afaire}{}{}
Acheter et laisser des \textbf{marqueurs indélébiles} pour que les gens écrivent leur surnom sur les tasses et ne les perdent pas.
\end{Afaire}

\begin{Asavoir}
Beaucoup de sites où il est possible de commander un grand nombre de tasses personnalisables ne tolèrent que les commandes venant d'une entreprise. Il faut en essayer plusieurs jusqu'à trouver le bon.
\end{Asavoir}

\begin{Attention}{}{}
Les tasses peuvent mettre longtemps à arriver. Il faut les commander au moins 1 mois à l'avance.
\end{Attention}

Entretenir \emph{un site internet} est très utile. Il permet de rappeler les informations essentielles et donner envie de venir. Notamment, il joue un rôle important dans la gestion des activités ($\to$ Activités \S\ref{sec:activités}).

L'option d'\emph{un Discord commun} sur lequel les participant·e·s peuvent communiquer est une très bonne idée. L'échange horizontal permet une auto-organisation pratique. De plus, les gens peuvent poser leurs questions directement aux organisateur·rice·s, ce qui est apprécié.

Sur place, \emph{des dépliants} (par exemple feuille A4 ou A5 pliée en deux) peuvent être distribués pour que chaque personne garde une trace des informations primordiales :
\begin{itemize}
    \item \textbf{Numéros d'urgence} et personnes à contacter
    \item \textbf{Heures des repas}
    \item \textbf{Modalités de sommeil} (et pédibus)
    \item \textbf{Plan du site}
    \item \textbf{Emploi du temps des activités}
\end{itemize}

Le top du top est d'avoir \textbf{un tableau blanc} dans la salle ou le couloir principal pour y afficher les modifications en temps réel. Notamment, certain·e·s voudront très probablement annoncer d'autres murders ou jdr en cours de weekend. Les gens peuvent s'y inscrire directement sur place.

%%%%%
\subsection{Repas}\label{sec:repas}

Il y a \emph{6 repas} à fournir :
\begin{itemize}
    \item \textbf{Vendredi diner} : peut être plus élaboré, mais doit être très rapide à réchauffer en grande quantité lorsque les participant·e·s des autres écoles arrivent en masse (ex. pâtes, riz, pois, ...)
    \item \textbf{Samedi petit-déjeuner}
    \item \textbf{Samedi déjeuner} : idéalement simple à préparer et pratique (car peu peuvent se mobiliser pour cuisiner le matin en s'étant couché·e·s tard), voire préparé à l'avance
    \item \textbf{Samedi diner} : quelque chose de facile à transporter dans les salles de jeux / murders / jdr, notamment plutôt à consistance ferme (ex.  pizza, tarte, burger, panini, crêpe, tacos, croques-monsieur,...)
    \item \textbf{Dimanche petit-déjeuner}
    \item \textbf{Dimanche déjeuner} : ne pas prévoir trop en quantité, car il serait dommage de jeter. C'est au contraire l'occasion de sortir les restes.
\end{itemize}

\begin{Afaire}{}{}
Notre conseil: un \textbf{buffet froid} pour les déjeuners. Les salades et tartes peuvent être préparées à l'avance et laissées au frigo. On peut aussi mettre à disposition du pain, des œufs durs, du pâté, du houmous, etc.
\end{Afaire}

Pour les \textbf{desserts}, prévoir simple et efficace : conserve, biscuits, compote, gâteaux préparé à l'avance (cookies, quatre-quart, ...), fruits, etc.

Pour les \emph{petits-déjeuners}, il faut essayer de prévoir de quoi satisfaire tout le monde, par exemple (quantités par personne):
\begin{itemize}
    \item \textbf{1 demi-baguette} ou 1/3 de baguette
    \item \textbf{du beurre\footnote{Salé et doux}, de la pâte à tartiner et de la confitures} (10g) : plusieurs saveurs si possible
    \item \textbf{1 jus de fruit} (20cl)
    \item \textbf{1 boisson chaude} (50cl) : café, thé et lait (+2 bouteille de lait végétal), mais tout le monde n'en boira pas
    \item \textbf{1 pain au chocolat ou croissant}
\end{itemize}

Niveau quantités, il faut prévoir des repas à environ \emph{200g à 300g par personne}, par exemple 100g de féculents, 50g de légumes et 80g de dessert. S'il y a des restes, les plus gourmand·e·s seront ravi·e·s du rab. Pour autant, il vaut mieux veiller, pendant le premier service, à servir tout le monde de manière équitable et selon les rations décidées pour être sûr que tout le monde ait un repas.

La \emph{quantité totale} peut aussi \textbf{varier selon les repas}. Il faut s'attendre à moins de petit-déjeuners (50\% des inscrit·e·s\footnote{Statistiques réalisées sur les interludes de 2019 à Cachulm}). Il y a des personnes (sur place par exemple) qui ne viennent que le samedi midi et soir (75\% des inscrit·e·s contre 60\% le vendredi soir et dimanche midi)\footnote{Attention, ces proportion varient d'une ENS à l'autre. Notamment, elles risquent d'être plus élevées à Rennes, où le peu de personnes vivant dans les alentours du campus incitent les gens à venir manger sur place.}.

Il faut compter \textbf{20\% de demandes végétariennes}. Ainsi, une option végétarienne, voire idéalement végétalienne, doit être prévue pour chaque repas. L'avantage de créer un menu végé un peu différent (et non juste sans viande), est d'offrir la possibilité d'une alternative pour celles et ceux qui n'aiment pas le plat principal. Pour les petits-déjeuners, il vaut mieux voir les demandes véganes au cas par cas (seulement 2 à 3 personnes souvent). Pour les \textbf{allergies} de même, préparer 1 ou 2 repas spéciaux est envisageable.

\phantomsection % fixes the problem of anchor and bookmark
%%%
\subsubsection{Buvette}

La buvette est un endroit où les personnes peuvent venir se restaurer entre les repas, notamment en pleine nuit ou le matin si une activité commence avant le petit-déjeuner. Il est courant d'offrir les boissons:
\begin{itemize}
    \item \textbf{thé} : 300 à 400 sachets (petit-déjeuners compris)
    \item \textbf{café} : 5kg (petit-déjeuners compris)
    \item \textbf{sirops} : 20L à 30L
\end{itemize}

Selon les moyens, des gâteaux peuvent aussi être mis en libre-service ou vendu à prix raisonnable. Les restes sucrés y partent aussi très rapidement. Par exemple, l'équivalent de 30 quatre-quarts réussit à partir entièrement en pratique.
%Corto : C'est plus pratique pour les orgas et les gens que tout soit gratuit lors du weekend, en particulier à la buvette. 

%%%
\subsubsection{Préparation}

Pour que tout le monde puisse aider, il est primordial que le ou la responsable des repas prépare une \textbf{fiche avec les recettes} et quantités de chacune.

\begin{Attention}{}{}
En cuisine, l'élément limitant est quasiment toujours \textbf{la cuisson}. Il est conseillé d'estimer la quantité maximale qui peut être cuite/réchauffée à la fois et évaluer la durée nécessaire pour anticiper. Quelques astuces pour plus d'efficacité:
\begin{itemize}
    \item Commencer à cuire les premiers aliments puis continuer l'épluchage/découpage des autres pendant ce temps
    \item Conserver l'eau chaude des pâtes et les récupérer à l'aide d'un grand écumoire
\end{itemize}
\end{Attention}


\begin{Aanticiper}{}{}
L'attente au comptoir de nourriture peut être longue et lassante. L'avantage du buffet est de laisser les gens se servir, pour qu'ils se ralentissent eux-mêmes. Le mieux reste de mettre en place une ligne de chaque côté de la table où sont les aliments.

Pour les plats servis par l'équipe, de même, le mot d'ordre est ``\textbf{paralléliser}'' : faire plusieurs files d'attentes en mobilisant tou·te·s les organisateur·rice·s à ce moment.
\end{Aanticiper}

Il existe des sets de couverts réutilisables, donc lavables entre les repas. Ça revient toujours à moins cher et c'est plus écologique. Par contre, il ne faut pas oublier de prévenir les gens.

%%%%%
\subsection{Budget et prix}

On vous propose ici une \emph{estimation du prix entrée seule} (pour 250 à 300 personnes):

\begin{tabular}{rrl}
Prix& total& $\approx$ par personne\\
     tasses &800--1200\euro&  3--5\euro\\
     vigiles (estimation \S\ref{sec:sécurité})&500\euro& 2\euro\\
     tombola& 50--150\euro& 0,4\euro\\
     affiches et dépliants& 70--150\euro& 0,4\euro\\
     bracelets et autres produits (ex. entretien)& 50--200\euro& 0,2--1\euro\\\hline
     \textbf{Total}& $\approx$ 1500--2200\euro &\textbf{6}--\textbf{9}\euro\\
     Total sans vigiles& $\approx$ 1000--1700\euro &4--7\euro
\end{tabular}

D'autres frais peuvent s'ajouter, notamment la location de salles
%($\approx$ 500\euro de location de gymnase ?)
ou d'équipement spécial éventuel.

Pour les repas, il est raisonnable de viser \emph{2\euro par repas par personne}, ce qui fait 12\euro les 6. Certains plats peuvent être plus chers en fonction des aliments, mais peuvent s'équilibrer avec d'autre moins chers. En comptant \textbf{entre 900 et 1000 repas} servis au total, cela revient à une dépense de 1800--2500\euro.

C'est au choix de l'équipe de choisir de mettre le prix de l'entrée plus élevé mais baisser celui des repas, ou l'inverse pour attirer les personnes sur place. On peut aussi, si le cout du l'hébergement est cher, répartir cette dépense pour le rendre plus accessible pour celles et ceux qui en ont absolument besoin.

%%%%%
\subsection{Inscriptions}

Les \emph{inscriptions} se font via un formulaire en ligne. Il permet de garder une trace des personnes qui sont censées venir, outre les gens sur place décidant de participer le jour j. Il donne aussi une estimation du nombre de personnes attendues, et donc surtout de la quantité de repas à fournir. Les informations essentielles à y demander sont:

\begin{itemize}
    \item \textbf{Prénom, nom}
    \item \textbf{Adresse email} (pour contacter)
    \item \textbf{Numéro de téléphone} (en cas d'urgence)
    \item \textbf{ENS de rattachement} (pour gérer les responsabilités et le paiement)
    \item \textbf{Nombre de repas prévus}
    \item \textbf{Régime alimentaire} (végétarien, allergies,...)
\end{itemize}

Étant donné les informations personnelles partagées, on déconseille un Google Form. On déconseille aussi LimeSurvey, peu pratique pour l'exportation des données. Notre conseil : \href{https://framaforms.org}{Framaform}.

\begin{Attention}{}{}
Ce formulaire ne se substitue pas à un formulaire d'inscription auprès de l'ENS de rattachement pour prévoir le trajet, que les participant·e·s devront remplir en plus.
\end{Attention}

Il n'y a pas de raison de fermer les inscriptions. Par contre, il est apprécié de pouvoir modifier le nombre de repas prévus, avant l'événement ou le jour d'arrivée. Cependant, il est en général refusé de compter le prix des repas après coup, en fonction de ceux pris effectivement pendant le weekend.

%%%%%
\subsection{Gestions des autres ENS}\label{sec:autres-ENS}

Les \emph{bureaux de gestion des événements ludiques respectifs à chaque ENS} sont l'intermédiaire entre l'équipe organisatrice et les participant·e·s. Leur rôle est de transmettre les informations et formulaires, prospecter les intentions de participation, répondre aux questions et collecter l'argent. Leur courriel sont:

% Dans l'ordre alphabétique
\newcommand{\email}[1]{\href{mailto:#1}{#1}}
\begin{itemize}
    \item \textbf{Lyon} : \email{bul@ens-lyon.fr}
    \item \textbf{Paris-Saclay} : \email{bdl-bureau@listes.crans.org}
    \item \textbf{Rennes} : \email{respo.ludes@listes.ens-rennes.fr}
    \item \textbf{Ulm} : \email{respojeux@ens.fr}
\end{itemize}

Un Discord commun peut aider à alléger ce maillon en mettant directement en relation l'équipe organisatrice et les participant·e·s. Mais pour l'équité, une confirmation de la transmission par courriel peut être demandée, notamment pour:

\begin{itemize}
    \item \textbf{la communication des dates}
    \item \textbf{le formulaire d'inscription}
    \item \textbf{le formulaire d'appel d'offre}
    \item \textbf{le formulaire d'inscription aux activités} (vœux)
\end{itemize}

Niveau argent, il est extrêmement désagréable pour les organisateur·rice·s de devoir traiter tout le monde séparément. C'est pourquoi il est impératif que les bureaux ludiques gèrent leurs affilié·e·s et rassemblent tout en une seule somme. Comme cette somme peut évoluer dans le temps en fonction des repas demandés par les participant·e·s, trois solutions sont possibles.

\begin{enumerate}
    \item Demander \textbf{dès la transmission du formulaire d'inscription}, une inscription supplémentaire à un \emph{autre formulaire} (de réponses identiques) créé par le bureau ludique pour savoir qui doit combien. Il peut aussi comprendre l'inscription au bus aller et retour. Mais il doit être mis à jour sur modification des repas.
    \item \emph{Partager le formulaire}\footnote{Pour Framaform, dans Modifier > Utilisateurs ayant accès aux résultats : si chercher le pseudonyme en fonctionne pas, ajouter le numéro d'utilisateur, accessible dans l'url de la page "Mon compte"} avec le bureau ludique
    \item \emph{Laisser l'orga calculer le montant après coup} et verser la somme par virement bancaire
\end{enumerate}

Un point à aborder avec les ENS distantes est le besoin ou non de sandwichs pour le trajet retour, et s'ils comptent (ou celui de l'aller), comme un repas supplémentaire aux frais de l'organisation.

%%%%%
\subsection{Activités}\label{sec:activités}

\phantomsection
%%%
\subsubsection{Récolter les projets}~

Traditionnellement, un appel à projet est lancé et chaque participant·e des Interludes est libre de proposer une \emph{activité}. La plupart de ces animations sont ludiques, mais on retrouve quelques représentations artistiques. Voici une proposition de classification de ce qui a déjà pu être proposé, avec le détail du nombre des activités proposées dans l'appel à projet aux Interludes de 2019 :

\begin{minipage}{0.5\textwidth}
\emph{Jeux}
\begin{itemize}
    \item jeux de rôles
    \begin{itemize}
        \item jdr sur table (5)
        \item murder / huis-clos (10)
        \item grandeur nature (0)
    \end{itemize}
    \item jeux d'ambiance (SporZ, Chaine alimentaire,...) (7)
    \item jeux d'enquête / énigmes (2)
    \item jeux de société (ex. initiation) (4)
    \item jeux de cartes (Koinche, Magic,...) (1)
    \item jeux vidéos (ex. tournoi) (2)
\end{itemize}
\end{minipage}
\begin{minipage}{0.5\textwidth}
\emph{Représentation artistique}
\begin{itemize}
    \item concert (Geekorale,...) (0)
    \item théâtre / impro (1)
\end{itemize}

\emph{Animations diverse}
\begin{itemize}
    \item fil rouge (0)
    \item tombola (1)
    \item autre (2)
\end{itemize}
\end{minipage}

Sur place, quelques jeux d'ambiances , jdr ou murders sont parfois proposés à l'improviste (compter environ 2 à 4 en tout en plus).

Étant donné la diversité des propositions et les diverses contraintes, il vaut mieux s'assurer dans l'\emph{appel à projet} d'avoir les informations nécessaires pour mener à bien chaque activité. Cette liste peut vous aider :

\begin{itemize}
    \item \textbf{informations personnelles} de l'organisateur·rice principal·e (nom, prénom, courriel, ENS de rattachement\footnote{Connaitre l'ENS de rattachement peut permettre d'anticiper les décalages d'activités dus au(x) retard(s) du vendredi soir})
    \item \textbf{titre}
    \item \textbf{type} de l'activité( cf. liste ci-haut)
    \item \textbf{nombre de participants} maximum (et nombre minimum)
    \item \textbf{type de participation} : inscription nécessaire à l'avance, inscriptions sur place ou pas d'inscription
    \item \textbf{description} de l'activité (ex. note d'intention incluse pour huis-clos)
    \item \textbf{temps estimé} \underline{briefing et débrief compris}
    \item \textbf{besoins (im)mobiliers} : nombre et tailles des salles, chaises, matériel technique (prise(s), projecteur,...)
    \item \textbf{disponibilité} des organisateur·rice·s
    \item \textbf{conflits potentiels} avec d'autres activités (car certaines personnes en organisent plusieurs)
    \item \textbf{remarques} : pour les demandes particulières, même si adresser un mail est souvent mieux
\end{itemize}

Pour la disponibilité des organisateur·rice·s, on conseille un tableau de cases à cocher organisé en tranches horaires (ex. vendredi soir, samedi matin,...) et appréciation de la disponibilité (absolument pas, plutôt pas, ok).

%%%
\subsubsection{Récolter les vœux et attribuer les inscriptions}~

\begin{Afaire}{}{}
Inscrire \textbf{les activité au fur et à mesure sur le site internet} permet aux gens de commencer à faire leurs choix de vœux. De plus, voir ce qui a été proposé peut en motiver certaines ou certains à se lancer, s'ils trouvent que ce la ne semble pas suffire pour contenter tout le monde. De même, une première version grossière de l'emploi du temps des animations sera appréciée.
\end{Afaire}

On préconise de fermer la collecte d'activités à inscription obligatoire trois semaines ($S$) avant l'événement et de suivre le déroulement suivant:
\begin{enumerate}
    \item $S-3$ : envoi du \emph{questionnaire de vœux}
    \item $S-2$ : fin (officielle) de la demande de vœux
    \item\label{it:attribution} $S-2 \,:\, S-1$ : \textbf{attribution des activité}
    \item $S-1$ : \textbf{communication des attributions}
    \item\label{it:roles} $S-1 \,:\, S$ (par les proposant·e·s) : envoi des questionnaires d'attribution de rôle, réponses des inscrit·e·s, attribution des rôles et envoi des fiches de personnage
\end{enumerate}

Cela signifie que l'\textbf{emploi du temps prévisionnel} doit idéalement être prêt et communiqué à l'ensemble des intéressé·e·s pour $S-3$. Pour cela, et puisque la répartition se base souvent au moins en partie sur la règle de rapidité (shotgun), il convient d'assurer une égalité des chances en annonçant le début des inscriptions à une date et heure précises, communiquée à tout le monde suffisamment de temps à l'avance.

Il faut bien compter une semaine pour l'étape \ref{it:roles}, étant donné le délai que peuvent prendre certaines personnes à répondre au questionnaire d'attribution des rôles.

\begin{Attention}{}{}
Il ne faut pas sous-estimer le temps que peut prendre l'étape \ref{it:attribution}. Même avec un bon algorithme d'attribution, il reste en général des \textbf{cas particuliers à régler à la main} (mauvais remplissage du formulaire, conflits d'emploi du temps, désistements, demandes spéciales,...). Cependant, les inscriptions peuvent rester ouvertes pendant ce temps.
\end{Attention}

La méthode d'\emph{attribution} ne met pas toujours tout le monde d'accord. Les arguments peuvent être ordonnés selon les principes des l'équipe organisatrice. Voici un exemple qui semble relativement raisonnable:

\begin{itemize}
    \item[++] \textbf{au moins une activité par demandeur·se}
    \item[+] \textbf{shotgun}
\end{itemize}

Ce choix est motivé par le principe de maximiser le bonheur total, en supposant qu'une personne ayant deux activités génère moins de bonheur que deux personnes ayant chacune une activité. Une adaptation de l'algorithme optimal du problème Hôpital-Patients (variante des \href{https://fr.wikipedia.org/wiki/Problème_des_mariages_stables}{mariages stables}) a été utilisé en 2019 avec beaucoup de succès. Cela demande notamment à chacun·e d'\textbf{ordonner ses vœux}, ce qui est selon nous une bonne pratique. L'idée est : d'abord attribuer tous les premiers vœux en commençant par les premièr·e·s à avoir répondu, et allant chercher le deuxième vœu ou plus si le premier est complet ; puis recommencer avec les deuxièmes vœux ou plus, etc.

\Tino{On espère rendre le code accessible et plus maniable d'ici peu.}

%%%%%
\subsection{Répartition des tâches et installation}

Le vendredi soir l'équipe est mobilisée pour \emph{installer}:
\begin{itemize}
    \item \textbf{récupérer les clés} des locaux
    \item \textbf{installer les tables et chaises} des salles de jeux en continu
    \item \textbf{apporter les jeux}
    \item \textbf{poser les panneaux} pour indiquer les salles
    \item \textbf{préparer le repas}
\end{itemize}

Il est courant de classer les jeux apportés par ENS d'origine pour ne pas les perdre, en installant des étiquettes bien visibles sur des tables disposées spécifiquement à cet effet.

Les volontaires de l'ENS d'accueil (aka. petits bras) sont d'une grande aide. C'est d'autant plus le cas au moment du \emph{rangement} le dimanche après-midi, car ranger et passer le balai partout, c'est long ! Par ailleurs, l'équipe est souvent trop restreinte pour assurer l'ensemble des permanences nécessaires :
\begin{itemize}
    \item \textbf{accueil} (au moins 3 files lors de l'arrivée de bus) : le vendredi soir et le samedi
    \item \textbf{buvette} (pour celles et ceux qui ont faim aux heures incongrues) : tout le temps
    \item \textbf{cuisine} : prévoir au moins 5 personnes 1 à 2h avant chaque repas
    \item (\textbf{sécurité} : tout le temps)
    \item \textbf{surveillance de nuit} : au moins deux personnes (et un jeu silencieux pour qu'elles ne s'ennuient pas)
\end{itemize}

\begin{Aanticiper}{}{}
Pour assurer un traitement rapide des problèmes, il est préférable que l'équipe se soit réparti les domaines de connaissances ($\to$ Équipe \S\ref{sec:equipe}) et que \textbf{les numéros de téléphones de ces responsables soient communiqués}. On peut par exemple afficher des fiches récapitulatives avec les numéros d'urgence aux endroits de permanence.
\end{Aanticiper}

On peux aussi envisager des bracelets (voire tichortes) spéciaux pour ces responsables.

%%%%%
\subsection{Chronologie prévisionnelle récapitulative}

En partant du principe que la date des Interludes est mi-férvier (jour $J$, semaine $S$) de l'année $A$, voici une chronologie suggestive pour visualiser les grandes étapes de l'organisation de ce rassemblement.

\hspace*{\fill}
\begin{tikzpicture}[thick, >=latex]
\draw[<->] (0,0) -- 
    node[left]{\begin{tabular}{c}avril -- mai\\$A-1$\end{tabular}}
    (0,-2);
\draw (4,-1) 
    node{\begin{tabular}{l}$\bullet$ \emph{création de l'équipe}\\$\bullet$ rédaction de la fiche technique\\$\bullet$ première \emph{réunion avec l'administration}\end{tabular}};
\draw[<->] (0,-2) -- 
    node[left]{juin -- aout}
    (0,-4);
\draw (3,-3) 
    node{\begin{tabular}{l}$\bullet$ temps libre pour anticiper:\\\quad ex. conception des visuels,\\\quad prévision des repas,...\end{tabular}};
\draw[<->] (0,-4) -- 
    node[left]{septembre -- octobre}
    (0,-6);
\draw (3.15,-5) 
    node{\begin{tabular}{l}$\bullet$ \emph{réunion avec l'administration}\\$\bullet$ \emph{communication des dates}\end{tabular}};
\draw[<->] (0,-6) -- 
    node[left]{novembre -- décembre}
    (0,-8);
\draw (3.1,-7) 
    node{\begin{tabular}{l}$\bullet$ \emph{formulaire d'inscription}\\$\bullet$ \emph{formulaire d'appel à projets}\end{tabular}};
\end{tikzpicture}
\hspace*{\fill}

\hspace*{\fill}
\begin{tikzpicture}[thick, >=latex]
\draw[-,semithick] (0,0) -- (0,-8);
\draw[<->] (0,0) -- 
    node[left]{\begin{tabular}{c}début janvier\\$A$\end{tabular}}
    (0,-1);
\draw (2.5,-0.5) 
    node{$\bullet$ \emph{commande des tasses}};
\draw[-] (-0.1,-2) node[left]{$S-3$} -- (0.1,-2);
\draw (3.2,-2) 
    node{\begin{tabular}{l}$\bullet$ emploi du temps prévisionnel\\$\bullet$ \emph{formulaire de vœux d'activités}\end{tabular}};
\draw[<->] (0,-3) -- 
    node[left]{$S-2$ -- $S-1$}
    (0,-4);
\draw (2.7,-3.5) 
    node{$\bullet$ \emph{attribution des activités}};
\draw[-] (-0.1,-4.5) node[left]{$S, J-5$} -- (0.1,-4.5);
\draw (3.05,-4.5) 
    node{\begin{tabular}{l}$\bullet$ \emph{courriel de rappel}\\[-3mm]\quad (penser au sac de couchage)\end{tabular}};
\draw[<->] (0,-5) -- 
    node[left]{$J-3$ -- $J-1$}
    (0,-5.5);
\draw (2.23,-5.25) 
    node{$\bullet$ gâteaux à l'avance};
\draw[-] (-0.1,-6) node[left]{$J$ (vendredi)} -- (0.1,-6);
\draw (1.73,-6) 
    node{\begin{tabular}{l}$\bullet$ \emph{courses}\\[-2mm]$\bullet$ \emph{installation} \end{tabular}};
\draw[dashed] (-0.07,-6) -- (-0.07,-6.5);
\draw[-] (-0.1,-7.5) node[left]{$S+1$} -- (0.1,-7.5);
\draw (1.3,-7.5) 
    node{$\bullet$ \emph{bilans}};
\end{tikzpicture}
\hspace*{\fill}

%%%%%%%%%%
\section{Préparation dans les ENS invitées}

%%%%%
\subsection{Préparation du voyage}

Les respos ludiques des ENS invitées doivent préparer le \emph{trajet commun} (sauf Ulm -- Paris-Saclay) de ses participants vers l'ENS accueillante, en car ou en train (+ bus). Pour anticiper les besoins, il est conseillé de distribuer un \emph{formulaire d'intention de participation aux Interludes} aux membres potentiellement intéressé·e·s dès l'annonce des dates. Cela permet notamment de savoir quelle taille de bus prendre (et pour Ulm -- Paris-Saclay, si un bus commun est envisageable).

Pour la gestion de l'argent, des solutions pratiques sont proposées à $\to$ Gestion des autres ENS \S\ref{sec:autres-ENS}. Il est vivement conseillé de faire circuler le \emph{formulaire d'inscription au bus} (en séparant aller du retour) en même temps que celui aux Interludes.

\textbf{Commander le bus} peut se faire via le site internet d'une compagnie de bus. Certaines proposent des devis gratuits. Compter une cinquantaine à soixantaine de places maximum par bus. Attention, c'est peut-être au BdE de payer si le club ludique dépend de lui. Les places peuvent rester cher, ce qui rend les Interludes moins abordables. À vous de voir si votre club est prêt à avancer de l'argent pour soulager les personnes non payées.

\begin{Aanticiper}{}{}
Pour que le bus entre dans l'enceinte de l'ENS avant le départ du vendredi et à l'arrivée du dimanche, il faudra sûrement \textbf{demander une autorisation}. Ne pas hésiter à faire la demande tôt, et à relancer.
\end{Aanticiper}

\begin{Afaire}{}{}
Comme il y a toujours des retardataires, il est bon de prévoir le \textbf{rendez-vous au moins une demi-heure avant le départ réel du bus}. Les retard jouent parfois sur le décalage des activités du vendredi soir, ce qui peut être désagréable. De plus, ne pas oublier une \textbf{fiche d'appel}, indispensable pour être sûr·e de n'avoir oublié personne sur les aires d'autoroute.
\end{Afaire}

Lors du voyage, il est apprécié de communiquer régulièrement une estimation du temps avant arrivée aux orgas. Anticiper les pauses se fait bien sûr aussi en discutant avec le chauffeur. Lors de celles-ci, vos participants apprécieront probablement quelques gâteaux ou friandises achetées à l'avance. Si le trajet fait plus de 6h, il faudra aussi des sandwichs, préparés à l'avance par vous ou les orgas: à voir ensemble.


%%%%%%%%%%
\section{Déroulement des Interludes}

%%%%%
\subsection{Début de semaine}

Tout temps gagné avant les Interludes sera une charge de moins le weekend fatidique. On conseille notamment de commencer à préparer les mets qui périment lentement (\textbf{gâteaux},...) avant le jour J. C'est aussi le moment de concevoir (et imprimer) les \textbf{pancartes d'orientation} et les \textbf{dépliants}.

%%%%%
\subsection{Vendredi}

Le vendredi matin est le bon moment pour faire \emph{les courses} pour tout le weekend. Il faut bien compter 4h voire 5h (rangement compris) à 3 personnes, vu le nombre d'ingrédients et les quantités nécessaires.

Le \textbf{repas du soir} peut commencer à être concocté dès le début d'après-midi, notamment si les salles ne sont disponible qu'en fin d'après-midi. Dès que celles-ci sont disponibles, des petits-bras peuvent aider à \textbf{ranger / déplacer les chaises et les tables} pour
\begin{itemize}
    \item installer l'espace jeux de sociétés / jeux vidéos
    \item installer l'accueil
    \item libérer les salles de huis-clos (et sommeil)
\end{itemize}

Dans le premières heures d'ouvertures (aux personnes de l'ENS d'accueil principalement), il est bien d'avoir au moins deux personnes à \emph{l'accueil}.

\begin{Aanticiper}{}{}
L'accueil a besoin d'une caisse \textbf{avec de la monnaie}. De même pour la buvette s'il y en a une payante de prévue.
\end{Aanticiper}

\begin{Attention}{}{}
Il est bon de \textbf{cocher les personnes qui sont effectivement venues} aux Interludes, notamment si on ne souhaite pas faire payer celles qui se sont inscrit mais finalement ne viennent pas. Cependant, certaines \textbf{peuvent ne pas être inscrites}. S'il suffit de rajouter des cases au tableur, lorsque celle-ci viennent des autres ENS, il ne faut pas oublier aussi de vérifier qui elles ont payé et / ou si elles doivent le faire le moment-même sur place.
\end{Attention}

Il est demandé aux responsables des ENS invitées de \textbf{prévenir l'équipe 10 minutes avant leur arrivée} en bus ou train. Ainsi, cette dernière a le temps d'installer d'autres files d'accueil pour ne pas y attendre trop longtemps. Le mot d'ordre : paralléliser.

Dans le cas (non pas rare) où un bus est en \textbf{retard}, il faut d'abord s'assurer de leur laisser assez de nourriture. Les activités organisées par ou ayant pour participant·e·s des personne·s de ce bus, sont décalées. Avoir un \textbf{tableau pour y afficher ces informations} est un grand atout. L'autre solution pour s'assurer que tout le monde est au courant est de recourir au mégaphone.

Après minuit\footnote{Le vendredi soir, c'est sûrement plutôt après 2h du matin que les premières personnes ont envie d'aller se coucher.}, certaines personnes peuvent demander à aller se coucher. Si les endroits où dormir ne sont pas à proximité, il faut démarrer les pédibus.

%%%%%
\subsection{Samedi}

À part vérifier que les organisateur·rice·s d'animations trouvent bien leur salle et leurs participant·e·s, il n'y a pas de grande demandes le samedi. \textbf{Servir les repas pendant les activités} est grandement apprécié.  Il est conseillé de toujours tenir l'accueil car certaines personnes ne viennent que le samedi. En dehors de la préparation et du service des repas, l'équipe peut profiter de quelques heures de détente bien méritées.

\begin{Afaire}{}{}
Un évènement avec des souvenirs, c'est mieux ! N'oubliez pas de prendre \emph{plein de photos} : jeux, repas, représentation, ... et de vous ;)
\end{Afaire}

%%%%%
\subsection{Dimanche}

En plus du repas du midi, il faudra peut-être préparer les sandwichs pour le(s) bus retour.

\emph{L'amphi de fin} marque la fin officielle du rassemblement. Il a souvent lieu entre 14h et 16h, pour finir avant que la première ENS ne parte. C'est l'occasion pour l'équipe de remercier les participant·e·s lors d'un petit discours, et de les inviter aux prochaines Interludes. Jusqu'à présent le cycle d'organisation est le suivant\footnote{Le nom d'une ENS entre deux années $m$ et $m+1$ correspond à l'année scolaire $m, m+1$ pendant laquelle elle organisera l'évènement, pour $n\in \mathbb{N}\setminus\{0\}$. Les deux premières fois étant en avril 2013 à Cachan puis en février 2014 à Lyon} :

\begin{center}
\begin{tikzpicture}[thick,  >=latex,]
\draw (0,0) node(c){Paris-Saclay};\draw (3,0) node(r){Rennes};
\draw (3,-1.5) node(u){Ulm};\draw (0,-1.5) node(l){Lyon};
\draw[->] (c) edge[bend left = 50] node[above]{$2011 + 4n$} (r);
\draw[->] (r) edge[bend left = 20] node[right]{$2012 + 4n$} (u);
\draw[->] (u) edge[bend left = 50] node[below]{$2013 + 4n$} (l);
\draw[->] (l) edge[bend left = 20] node[left]{$2010 + 4n$} (c);
\end{tikzpicture}
\end{center}


Lors de la \emph{tombola}, il est intéressant de proposer des jeux ou livres divers, par exemple d'une valeur totale de 50\euro à 150\euro. La méthode d'attribution et de sélection des tickets gagnant est souvent débattue, et plusieurs ont déjà été essayées : distribution de ticket(s) aux personnes assises à l'amphi, à chaque activité réalisée au cours du weekend, pioche dans un chapeau, avec des dés,... On partage ici un système facile à mettre en place et sûr :

\begin{itemize}
    \item Imprimer environ 100 \textbf{papiers contenant deux fois le même numéro}.
    \item À l'entrée de chaque personne dans l'amphi, prendre un papier, \textbf{le couper en deux}, donner un numéro à la personne et mettre \textbf{l'autre dans le chapeau}
\end{itemize}

Ce procédé, bien que plus lent à la mise en place, a l'énorme avantage de ne jamais faire piocher de numéro non attribué, et ainsi perdre moins de temps.

% \begin{Aanticiper}{}{}
% Mieux vaut prévoir d'imprimer maximum 150 tickets de tombola à l'avance, ça peut prendre un peu de temps à découper.
% \end{Aanticiper}

Le temps que les bus arrivent, les gens peuvent reprendre un jeu de société. Mais il faut leur annoncer d ne pas en commencer un long 30 minutes avant le départ. Les chef·fe·s des ENS invitées peuvent demander le mégaphone pour appeler leurs rattaché·e·s.

%%%%%%%%%%
\section{Après les Interludes}

%%%%%
\subsection{Bilans}

Un \textbf{courriel de remerciement} est toujours apprécié. C'est aussi l'occasion d'y partager les photos prises lors du rassemblement. Quelques pertes (effets personnels, jetons de jeu) peuvent être déclarées.

Le plus tôt possible, idéalement \textbf{une semaine après}, une réunion d'équipe permet de mettre à plat comment se sont déroulées les Interludes, ce qui a bien fonctionné et ce qui a moins bien marché (en reprenant par exemple les points de cette fiche). C'est le \emph{bilan technique}. Un bilan financier est aussi attendu. Selon les envies, un formulaire de contentement peut être envoyé (compter entre 30 et 50 réponses). L'événement se déroule généralement sans incident majeur et la plupart des participant·e·s sont (très) satisfaits.

Mais il est aussi important d'aborder le \emph{bilan moral}. Ici, il est question de dire franchement comment chaque personne a ressenti cette épreuve. Notamment s'il y a eu des tensions, et comment elles ont été résolues. Il s'agit avant tout d'une aventure humaine, qui nous fait apprendre à toutes et à tous comment réaliser de grandes choses ensemble.

\end{document}
